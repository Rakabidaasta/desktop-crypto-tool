%%% Для сборки выполнить 2 раза команду: pdflatex <имя файла>

\documentclass[a4paper,12pt]{article}

\usepackage{ucs}
\usepackage[utf8x]{inputenc}
\usepackage[russian]{babel}
%\usepackage{cmlgc}
\usepackage{graphicx}
\usepackage{hyperref}
\usepackage{listings}
\usepackage{xcolor}
%\usepackage{courier}

\makeatletter
\renewcommand\@biblabel[1]{#1.}
\makeatother

\newcommand{\myrule}[1]{\rule{#1}{0.4pt}}
\newcommand{\sign}[2][~]{{\small\myrule{#2}\\[-0.7em]\makebox[#2]{\it #1}}}

% Поля
\usepackage[top=20mm, left=30mm, right=10mm, bottom=20mm, nohead]{geometry}
\usepackage{indentfirst}

% Межстрочный интервал
\renewcommand{\baselinestretch}{1.50}


\begin{document}

%%%%%%%%%%%%%%%%%%%%%%%%%%%%%%%
%%%                         %%%
%%% Начало титульного листа %%%

\thispagestyle{empty}
\begin{center}


\renewcommand{\baselinestretch}{1}
{\large
{\sc Петрозаводский государственный университет\\
Институт математики и информационных технологий\\
	Кафедра Информатики и Математического Обеспечения
}
}

\end{center}


\begin{center}
%%%%%%%%%%%%%%%%%%%%%%%%%
%
% Раскомментируйте (уберите знак процента в начале строки)
% для одной из строк типа направления  - бакалавриат/
% магистратура и для одной из
% строк Вашего направление подготовки
%
% Направление подготовки бакалавриата \\
% 01.03.02 Прикладная математика и информатика \\
% 09.03.02 - Информационные системы и технологии \\
09.03.04 - Программная инженерия \\
% Направление подготовки магистратуры \\
% 01.04.02 - Прикладная математика и информатика \\
% 09.04.02 - Информационные системы и  технологии \\
%
% 
%%%%%%%%%%%%%%%%%%%%%%%%%
	% \textcolor{red}{<Ваши тип и направление подготовки>} 
\end{center}

\vfill

\begin{center}
{\normalsize Отчет о проектной работе по курсу <<Основы информатики и программирования>>} \\

\medskip

%%% Название работы %%%
	{\Large \sc Приложение <<Desktop crypto tool>>} \\
	% (промежуточный)
\end{center}

\medskip

\begin{flushright}
\parbox{11cm}{%
\renewcommand{\baselinestretch}{1.2}
\normalsize
	Выполнила:\\
%%% ФИО студента %%%
студентка 1 курса группы 22107
\begin{flushright}
	Т. Д. Квист \sign[подпись]{4cm}
\end{flushright}

Руководитель:\\
А. В. Бородин, старший преподаватель \\
% \begin{flushright}
% \sign[подпись]{4cm}
% \end{flushright}

}
\end{flushright}

\vfill

\begin{center}
\large
    Петрозаводск --- 2021
\end{center}

%%% Конец титульного листа  %%%
%%%                         %%%
%%%%%%%%%%%%%%%%%%%%%%%%%%%%%%%

%%%%%%%%%%%%%%%%%%%%%%%%%%%%%%%%
%%%                          %%%
%%% Содержание               %%%

\newpage

\hypersetup{hidelinks}
\tableofcontents

\newpage
\section*{Введение}
\addcontentsline{toc}{section}{Введение}


Цель проекта: разработать приложение для решения основных криптографических задач. \\

Задачи проекта: 
\begin{enumerate}
    \item Разработать модуль для подбора строки, которой соответствует заданный хеш.
    \item Разработать модуль для дешифровки RSA.
    \item Разработать модуль для перевода кодировки из UTF-8 в base16, base32 и base64.
    \item Разработать графический интерфейс пользователя.
    \item Реализовать приложение с использованием разработанных модулей и QtQuick.
\end{enumerate}

В мире криптографии часто приходится решать какие-то задачи, которые человек может выполнять несколько часов, а то и дней. Для этого создаются приложения, автоматизирующие эти процессы: вычисление сложных математических операций, brute-force (метод перебора грубой силы) и т.п. Случаются ситуации, когда нет доступа к сети Интернет, в связи с этим создаются оффлайн приложения. Основная цель этого проекта: разработать оффлайн приложение, которое поможет решить некоторые базовые задачи криптографии (подбор хеша, дешифровка RSA, перевод из одной кодировки в другую). Для достижения этой цели необходимо разработать соответствующие модули. 

%%%                          %%%
%%%%%%%%%%%%%%%%%%%%%%%%%%%%%%%%

\newpage

%%%%%%%%%%%%%%%%%%%%%%%%%%%%%%%%
%%%                          %%%
%%% Требования к приложению  %%%

\section{Требования к приложению}
\begin{itemize}
    \item Подбор MD5 хеша методом грубой силы.
    \item Дешифровка криптографического алгоритма RSA.
    \item Перевод строки из UTF-8 в base16 (другими словами hex), base32, base64. 
\end{itemize}

%%%                          %%%
%%%%%%%%%%%%%%%%%%%%%%%%%%%%%%%%

\newpage

%%%%%%%%%%%%%%%%%%%%%%%%%%%%%%%%%
%%%                           %%%
%%% Проектирование приложения %%%
\section{Проектирование приложения}
Модули приложения:
\begin{enumerate}
    \item hash.cpp --- работа с хешем. Основная функция модуля:
    \begin{itemize}
        \item check\_hash() --- проверка наличия строки в словаре (rockyou.txt), для которой хеш будет совпадать с заданной строкой.
    \end{itemize}
    \item rsa.cpp --- работа с RSA. Основные функции модуля:
    \begin{itemize}
        \item solve\_rsa() --- подготовка всех необходимых переменных для дешифровки RSA.
        \item calculateD() --- подсчёт числа d, для которого будет выполняться следующее условие: $d \cdot e = 1\; mod \; phi$
        \item decrypt() --- дешифровка сообщения со всеми необходимыми переменными.
    \end{itemize}
    \item bases.cpp --- работа с кодировками. Основная функция модуля:
    \begin{itemize}
        \item bases\_encode() --- перевод строки из UTF-8 в следующие кодировки: base16, base32 и base64. Используется сторонняя библиотека для подсчёта base32 и base64: \url{https://github.com/tplgy/cppcodec}.
    \end{itemize}
    \item Page1Form.ui.qml --- графический интерфейс главной страницы.
    \item Page2Form.ui.qml --- графический интерфейс для работы с хешами.
    \item Page3Form.ui.qml --- графический интерфейс для работы с RSA.
    \item Page4Form.ui.qml --- графический интерфейс для работы с кодировками.
    \item main.qml --- главный модуль графического интерфейсаю
    \item main.cpp --- главный модуль для работы с фунциями на языке <<C++>>, в котором инициализируются экземпляры классов Hash, RSA и Bases.
\end{enumerate}

%%%                          %%%
%%%%%%%%%%%%%%%%%%%%%%%%%%%%%%%%

\newpage

%%%%%%%%%%%%%%%%%%%%%%%%%%%%%%%%%
%%%                           %%%
%%% Реализация приложения     %%%
\section{Релизация приложения}
Для реализации приложения были использованы языки <<C++>> и <<QML>>. 
\begin{itemize}
    \item Количество модулей: 5.
    \item Количество классов: 3.
    \item Количество <<C++>> функций: 5.
    \item Количество <<QML>> сигналов: 6.
    \item Количество строк <<C++>> кода: 227.
    \item Количество строк <<QML>> кода: 661.
\end{itemize}

%%%                          %%%
%%%%%%%%%%%%%%%%%%%%%%%%%%%%%%%%

\newpage

%%%%%%%%%%%%%%%%%%%%%%%%%%%%%%%%%
%%%                           %%%
%%% Заключение                %%%

\section*{Заключение}
\addcontentsline{toc}{section}{Заключение}

В результате проекта было разработано приложение для решения основных криптографических задач. Пользователь может узнать для какой строке соответствует заданный хеш, дешифровать RSA и поменять кодировку у заданной строки.\\

Получен опыт работы с криптографическими библиотеками языка <<C++>>, а также опыт работы с <<QtQuick>>.


\end{document}
