\documentclass{article}
\usepackage{geometry}
\usepackage[T2A]{fontenc}
\usepackage[utf8]{inputenc}
\usepackage[english, russian]{babel}

\geometry{bmargin = 80pt, lmargin = 80pt, tmargin = 80pt, rmargin = 80pt}

\begin{document}
\section{Тестирование hash}
\subsection{default}
Входные данные -- MD5-хеш строки \textit{hello}, то есть \textit{5d41402abc4b2a76b9719d911017c592}.\\
Ожидаемый результат: 
\begin{itemize}
    \item res == "hello". 
\end{itemize}

\subsection{short\_hash}
Входные данные -- строка длиной в 31 символ, что меньше, чем длина MD5-хеша.\\
Ожидаемый результат: 
\begin{itemize}
    \item res == "Это не MD5 хэш!". 
\end{itemize}

\subsection{long\_hash}
Входные данные -- строка длиной в 33 символа, что больше, чем длина MD5-хеша.\\
Ожидаемый результат: 
\begin{itemize}
    \item res == "Это не MD5 хэш!". 
\end{itemize}

\subsection{nothing\_find}
Входные данные -- MD5-хеш строки, которой нет в словаре (rockyou.txt).\\
Ожидаемый результат: 
\begin{itemize}
    \item res == "Ничего не нашлось :(". 
\end{itemize}

\section{Тестирование rsa}
\subsection{prime\_p}
Входные данные -- p = 62, q = 53, e = 17, ct = 3. p --- непростое число.\\
Ожидаемый результат: 
\begin{itemize}
    \item res == -1. 
\end{itemize}

\subsection{prime\_q}
Входные данные -- p = 61, q = 52, e = 17, ct = 3. q --- непростое число.\\
Ожидаемый результат: 
\begin{itemize}
    \item res == -1. 
\end{itemize}

\subsection{prime\_e}
Входные данные -- p = 61, q = 53, e = 10, ct = 3. e --- непростое число.\\
Ожидаемый результат: 
\begin{itemize}
    \item res == -1. 
\end{itemize}

\subsection{example}
Входные данные -- p = 61, q = 53, e = 17, ct = 2790. Пример из Википедии.\\
Ожидаемый результат: 
\begin{itemize}
    \item res == 65. 
\end{itemize}

\subsection{zero}
Входные данные -- p = 0, q = 0, e = 0, ct = 10. Проверка на нулевые аргументы.\\
Ожидаемый результат: 
\begin{itemize}
    \item res == -1. 
\end{itemize}

\subsection{prime}
Входные данные -- n = 10, n = 11, n = -10. Проверка на простые числа. Функция IsPrime.\\
Ожидаемый результат: 
\begin{itemize}
    \item res == false, res2 == true, res3 == false 
\end{itemize}

\subsection{coprime\_d}
Входные данные -- e = 6, phi = 10. Проверка на расчёт сопростого числа. Функция calculateD.\\
Ожидаемый результат: 
\begin{itemize}
    \item res == -1 
\end{itemize}

\subsection{coprime\_d\_zero}
Входные данные -- e = 0, phi = -10. Проверка на расчёт сопростого числа. Функция calculateD.\\
Ожидаемый результат: 
\begin{itemize}
    \item res == -1 
\end{itemize}

\section{Тестирование bases}
\subsection{example}
Входные данные -- строка \textit{lalala}.\\
Ожидаемый результат: 
\begin{itemize}
    \item res[0] == "6C616C616C61"
    \item res[1] == "NRQWYYLMME======"
    \item res[2] == "bGFsYWxh" 
\end{itemize}

\subsection{clipboard}
Входные данные -- в буфер помещается строка "hello".\\
Ожидаемый результат: в буфере находится строка "hello", get\_from\_clipboard == "hello"


\end{document}